% ============================================================
%  INTRODUCCIÓN
% ============================================================
\chapter{Introducción}

\section{Contexto y motivación}

Las bases de datos están en todas partes, pero es perfectamente posible usarlas durante años sin entender bien cómo funcionan por dentro. Cuando se ejecuta un \texttt{COMMIT}, algo pasa: alguien escribe en disco, alguien libera bloqueos, alguien garantiza que esos cambios sobrevivan a un corte de luz. Ese ``alguien'' es la arquitectura interna del gestor, y entenderla es el propósito de este trabajo.

Para entender cómo están diseñados estos sistemas hoy, hay que saber por qué tomaron las decisiones que tomaron. Y muchas de esas decisiones tienen una explicación histórica y técnica que empieza mucho antes de que existiera Oracle o PostgreSQL.

\section{Planteamiento del problema}

La pregunta que guía esta investigación es: \textit{¿cómo condicionó la evolución del hardware la adopción del modelo relacional, y cómo se refleja eso en la arquitectura de los gestores actuales?}

Codd publicó su modelo en 1970, pero durante más de una década no fue adoptado masivamente. La razón no fue teórica: el modelo relacional exige operaciones intensivas en memoria, y en los años 70 esa memoria era cara y escasa. Cada componente de un gestor moderno, el buffer cache, el redo log, la PGA, existe como respuesta concreta a ese problema.

\section{Alcances y limitaciones}

Este trabajo es documental y comparativo. Se analiza la arquitectura de Oracle, PostgreSQL, MySQL con InnoDB y SQL Server, con foco en memoria, procesos, archivos físicos y logging transaccional. No se realizaron pruebas de rendimiento ni instalaciones reales.

\section{Estructura del documento}

El \textbf{Capítulo 2} presenta el marco teórico. El \textbf{Capítulo 3} cubre la historia desde los archivos planos hasta los primeros gestores comerciales. El \textbf{Capítulo 4} explica cómo el hardware hizo viable el modelo relacional. El \textbf{Capítulo 5} describe la arquitectura física de Oracle. El \textbf{Capítulo 6} compara los cuatro motores. Al final se presentan conclusiones y recomendaciones.
\clearpage