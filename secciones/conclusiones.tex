
%%%%%%%%%%%%%%%%%%%%%%%%%%%%%%%%%%%%%%%%%%%%%%%%%%%%%%%%%
% SECCIÓN 7: CONCLUSIONES (VERSIÓN RESUMIDA Y HUMANIZADA)
%%%%%%%%%%%%%%%%%%%%%%%%%%%%%%%%%%%%%%%%%%%%%%%%%%%%%%%%%

\section{Conclusiones}

Esta investigación permitió entender que las bases de datos no surgieron de la nada: fueron evolucionando conforme el hardware y las necesidades lo fueron permitiendo. A continuación se resumen las ideas más importantes:

\begin{enumerate}
    \item \textbf{El hardware limitó la teoría:} 
    Codd propuso el modelo relacional en 1970, pero tardó más de una década en adoptarse masivamente porque la memoria RAM era carísima. El modelo era bueno desde el principio, el hardware era el que no estaba listo.

    \item \textbf{Pasar de archivos planos a tablas fue un salto enorme:} 
    Los sistemas anteriores mezclaban la lógica de los datos con cómo estaban guardados físicamente. Separar esas dos cosas fue lo que hizo que las bases de datos relacionales fueran tan poderosas y flexibles.

    \item \textbf{Oracle es lo que es gracias a años de experiencia:} 
    Cada proceso y componente de su arquitectura existe para responder una pregunta concreta: ¿cómo no perder datos si se va la luz? ¿cómo atender a miles de usuarios a la vez sin que se pisen? No es diseño caprichoso, es experiencia acumulada.


    \item \textbf{Garantía ACID en acción:} 
    La integridad de la información surje gracial al trabajo coordinado entre memoria, procesos y archivos. Elementos como el \textit{buffer cache} y los \textit{redo logs} son los que permiten que las propiedades ACID pasen de ser conceptos teóricos a elementos reales de seguridad para el usuario.

    \item \textbf{Diversidad de soluciones:} 
    La comparativa técnica revela que no existe un SGBD "perfecto", sino uno adecuado para cada necesidad. Mientras Oracle y SQL Server lideran el sector empresarial, PostgreSQL destaca por su robustez de código abierto y MySQL por su agilidad en entornos web de alta lectura.

    \item \textbf{La arquitectura dicta el éxito:} 
    Finalmente, se concluye que el rendimiento y la escalabilidad de cualquier sistema de información dependen de su base física. La elección del SGBD correcto debe basarse en un análisis crítico de la carga de trabajo, el presupuesto y las capacidades técnicas del equipo.
\end{enumerate}
\clearpage