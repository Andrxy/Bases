% ============================================================
%  MARCO TEÓRICO
% ============================================================
\chapter{Marco teórico}

% TODO: El marco teórico establece las bases conceptuales sobre las que se apoya
%       el desarrollo temático. Debe fundamentarse en fuentes académicas (APA 7).


\section{Definición de base de datos y SGBD}

Una base de datos es una colección organizada de datos relacionados que representan aspectos del mundo real, mientras que un SGBD es el software que permite su creación, mantenimiento y control de acceso, actuando como interfaz entre el usuario y los archivos físicos. Diversos autores han aportado definiciones complementarias:

\begin{itemize}
    \item \textbf{C.J. Date} la describe como un sistema para almacenar información y permitir su disponibilidad bajo demanda.
    \item \textbf{Silberschatz, Korth y Sudarshan} la definen como una colección de datos interrelacionados junto con programas para acceder a ellos.
    \item \textbf{Edgar F. Codd} enfocó su definición en la independencia de los datos respecto al almacenamiento físico, permitiendo interactuar con una representación lógica de la información.
\end{itemize}

\section{Modelo relacional}

Propuesto por \textcite{codd_1970} , organiza los datos mediante \textbf{relaciones} (tablas), \textbf{tuplas} (filas) y \textbf{atributos} (columnas), garantizando la \textbf{integridad referencial} entre tablas. Complementariamente, los \textbf{doce principios de Codd} establecen que un sistema debe gestionar la base de datos exclusivamente a través de sus capacidades relacionales, asegurando la independencia física y lógica de los datos.

\section{Propiedades ACID}

Las propiedades ACID garantizan la fiabilidad de las transacciones en los SGBD de producción:

\begin{itemize}
    \item \textbf{Atomicidad:} una transacción se completa en su totalidad o no se ejecuta en absoluto.
    \item \textbf{Consistencia:} cada transacción lleva la base de datos de un estado válido a otro, respetando las reglas de integridad.
    \item \textbf{Aislamiento:} las transacciones se ejecutan de forma independiente, sin interferencias entre operaciones simultáneas.
    \item \textbf{Durabilidad:} los efectos de una transacción confirmada son permanentes, incluso ante fallos del sistema.
\end{itemize}

\section{Arquitectura cliente-servidor en bases de datos}

Define cómo los clientes interactúan con el motor de base de datos. Sus principales variantes son:

\begin{itemize}
    \item \textbf{2 capas:} el cliente se comunica directamente con el servidor de base de datos.
    \item \textbf{3 capas:} un servidor de aplicaciones intermedio mejora la escalabilidad y la seguridad.
    \item \textbf{Distribuida:} la base de datos se reparte entre múltiples nodos, aumentando la disponibilidad y el procesamiento paralelo.
\end{itemize}

En sistemas como Oracle, esta arquitectura gestiona procesos en segundo plano (como DBWR y LGWR) para coordinar las peticiones concurrentes en entornos multiusuario.

\section{Conceptos de memoria y almacenamiento relevantes}

El rendimiento de un SGBD depende estrechamente de su arquitectura de memoria y almacenamiento:

\begin{itemize}
    \item \textbf{RAM:} permite mantener buffers amplios que reducen los accesos al almacenamiento físico, impactando directamente en el rendimiento.
    \item \textbf{Caché y Buffer Pool:} almacenan temporalmente los datos más consultados. En Oracle se denomina \textit{Buffer Cache}, dentro de la SGA.
    \item \textbf{Almacenamiento secundario (HDD/SSD):} aloja de forma permanente los datafiles y archivos de control. La adopción de SSD ha mejorado drásticamente las velocidades de lectura y escritura.
\end{itemize}

La arquitectura de memoria de Oracle (SGA y PGA) condiciona la capacidad del sistema para gestionar transacciones complejas y controlar la concurrencia eficientemente.
\clearpage