% ============================================================
%  INTRODUCCIÓN
% ============================================================
\chapter{Introducción}

\section{Contexto y motivación}

Hoy casi cualquier sistema de información depende de una base de datos, pero es fácil usarlas sin entender cómo funcionan por dentro. Saber qué ocurre exactamente cuando se ejecuta un \texttt{COMMIT}, quién escribe los datos al disco o qué pasa ante una falla de energía no suele cubrirse en cursos introductorios. Esta investigación parte de esa curiosidad: no solo describir los sistemas, sino entender por qué están diseñados como están, algo que requirió ir hacia atrás y revisar el contexto histórico y de hardware en que surgieron.

\section{Planteamiento del problema}

La pregunta central es: \textit{¿cómo condicionó la evolución del hardware la adopción masiva del modelo relacional, y cómo se refleja eso en la arquitectura física de los SGBD actuales?}

Codd publicó su propuesta en 1970 pero durante más de una década fue considerada demasiado costosa para uso práctico. La razón es concreta: el modelo relacional exige operaciones intensivas en memoria, y en los años 70 esa memoria era escasa y cara. Cada componente de la arquitectura de un SGBD moderno existe, en parte, como respuesta a esa limitación histórica.

\section{Alcances y limitaciones}

El análisis cubre Oracle, PostgreSQL, MySQL con InnoDB y SQL Server, enfocándose en arquitectura de memoria, procesos, archivos físicos y logging transaccional. No se realizaron pruebas empíricas de rendimiento ni instalaciones reales; el trabajo es documental y comparativo.

\section{Estructura del documento}

El \textbf{Capítulo 2} establece el marco teórico. El \textbf{Capítulo 3} recorre la historia desde los archivos planos hasta los primeros SGBD comerciales. El \textbf{Capítulo 4} analiza cómo el hardware hizo viable el modelo relacional. El \textbf{Capítulo 5} describe la arquitectura física de Oracle. El \textbf{Capítulo 6} presenta la comparativa entre los cuatro motores. Finalmente se presentan conclusiones y recomendaciones.
\clearpage