% ============================================================
%  RECOMENDACIONES
% ============================================================
\chapter{Recomendaciones}

\begin{enumerate}[leftmargin=*, label=\arabic*.]

  \item \textbf{Antes de elegir un gestor, entender qué se necesita.} No es lo mismo un sistema para una tienda en línea que para un banco. Definir primero el volumen de usuarios, la criticidad de los datos y el presupuesto disponible evita decisiones que después son costosas de corregir.

  \item \textbf{No operar Oracle sin entender sus procesos de fondo.} Modificar parámetros como el tamaño del buffer cache o los grupos de redo log sin conocer qué hace DBWR o LGWR puede empeorar el rendimiento en lugar de mejorarlo. La documentación oficial de Oracle es extensa pero necesaria.

  \item \textbf{Multiplicar los archivos de control en Oracle.} Mantener copias del control file en discos físicos distintos es una de las medidas de protección más simples y más ignoradas. Sin esas copias, una falla de disco puede dejar la instancia irrecuperable.

  \item \textbf{Separar los tablespaces según la naturaleza de los datos.} Tener tablespaces distintos para datos, índices, temporales y sistema no es solo orden: facilita el monitoreo del espacio, el respaldo selectivo y la recuperación ante fallos parciales.

\end{enumerate}
\clearpage