% ============================================================
%  MARCO TEÓRICO
% ============================================================
\chapter{Marco teórico}

\section{Definición de base de datos y SGBD}

Una base de datos es una colección organizada de datos que representa aspectos del mundo real. Un SGBD es el software que permite crear, mantener y controlar el acceso a esos datos, actuando como intermediario entre el usuario y los archivos físicos en disco.

La distinción más relevante para este trabajo viene de \textcite{codd_1970}: separar la visión lógica de los datos de cómo están físicamente almacenados. Esa independencia fue la ruptura central que hizo posible el modelo relacional y es lo que diferencia un SGBD moderno de un simple sistema de archivos.

\section{Modelo relacional}

La propuesta de \textcite{codd_1970} organizó los datos en \textbf{relaciones} (tablas), \textbf{tuplas} (filas) y \textbf{atributos} (columnas), apoyada en álgebra relacional. Lo que la distinguió de los modelos anteriores no fue solo la estructura tabular, sino la independencia física: la misma consulta funcionaba sin importar cómo estuvieran organizados los datos en disco. El usuario no necesitaba saber nada de la implementación interna.

\section{Propiedades ACID}

Las propiedades ACID definen qué significa que una transacción sea confiable \parencite{haerder1983}:

\begin{itemize}
    \item \textbf{Atomicidad:} la transacción se ejecuta completa o no se ejecuta.
    \item \textbf{Consistencia:} lleva la base de datos de un estado válido a otro.
    \item \textbf{Aislamiento:} las transacciones concurrentes no se interfieren entre sí.
    \item \textbf{Durabilidad:} los efectos de una transacción confirmada persisten ante fallos.
\end{itemize}

Estas cuatro propiedades no son solo conceptos teóricos: cada componente físico de la arquitectura de un SGBD existe, en parte, para implementarlas.

\section{Arquitectura cliente-servidor}

La separación entre cliente y servidor permite que múltiples usuarios trabajen simultáneamente sobre la misma base de datos. Oracle añade una distinción importante: la \textbf{instancia} (memoria y procesos, temporal) frente a la \textbf{base de datos} (archivos en disco, persistente). Ante una falla, la instancia desaparece pero los archivos permanecen, y eso es lo que hace posible la recuperación.

\section{Memoria y almacenamiento}

El rendimiento de un SGBD depende en gran medida de cuánto puede mantener en RAM y cuánto debe ir a buscar al disco. La diferencia de velocidad entre ambos es de varios órdenes de magnitud, y eso explica por qué el dimensionamiento del buffer cache es una de las decisiones más críticas en la administración de cualquier motor.

\clearpage