% ============================================================
%  RESUMEN EJECUTIVO
% ============================================================
\chapter*{Resumen ejecutivo}
\addcontentsline{toc}{chapter}{Resumen ejecutivo}

Este trabajo analiza cómo la evolución del hardware condicionó la adopción de los gestores de bases de datos relacionales, con énfasis en la arquitectura de Oracle Database y su comparación con PostgreSQL, MySQL (InnoDB) y Microsoft SQL Server.

El modelo relacional de Codd tardó más de una década en adoptarse masivamente, y la razón fue concreta: la RAM era cara y escasa, y el modelo la necesitaba. Cuando el hardware maduró, el modelo se impuso rápidamente.

El trabajo describe la arquitectura interna de Oracle (memoria, procesos en segundo plano, archivos físicos y mecanismos ACID) y contrasta sus decisiones de diseño con las de los otros tres motores. La conclusión principal no es que haya un gestor superior a los demás, sino que cada uno resolvió los mismos problemas desde enfoques distintos, con ventajas y costos propios.

\bigskip
\noindent\textbf{Palabras clave:} bases de datos relacionales, Oracle Database, arquitectura SGBD, PostgreSQL, MySQL, SQL Server, Edgar F. Codd.
\clearpage