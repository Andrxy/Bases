% ============================================================
%  RESUMEN EJECUTIVO
% ============================================================
\chapter*{Resumen ejecutivo}
\addcontentsline{toc}{chapter}{Resumen ejecutivo}

Este trabajo examina cómo la evolución del hardware condicionó la adopción masiva de los sistemas gestores de bases de datos relacionales, con énfasis en la arquitectura de Oracle Database y su comparación con PostgreSQL, MySQL (InnoDB) y Microsoft SQL Server. La demora en la adopción del modelo relacional de Codd respondió menos a limitaciones teóricas que a restricciones concretas de memoria y procesamiento: la RAM era escasa y cara, y el modelo relacional la necesitaba. Cuando el hardware maduró, el modelo se impuso con rapidez.

El análisis describe la arquitectura interna de Oracle (memoria, procesos en segundo plano, archivos físicos y mecanismos ACID) y contrasta sus decisiones de diseño con las de los otros tres motores. La conclusión no es que haya un sistema superior, sino que cada uno resolvió los mismos problemas desde filosofías distintas, con ventajas y costos propios. El trabajo se basa en documentación oficial y bibliografía académica, no incluye pruebas empíricas de rendimiento.

\bigskip
\noindent\textbf{Palabras clave:} bases de datos relacionales, Oracle Database,
arquitectura SGBD, memoria RAM, PostgreSQL, MySQL, SQL Server, Edgar F. Codd.
\clearpage