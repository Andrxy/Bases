% ============================================================
%  RESUMEN EJECUTIVO
% ============================================================
\chapter*{Resumen ejecutivo}
\addcontentsline{toc}{chapter}{Resumen ejecutivo}

Este trabajo examina cómo la evolución del hardware condicionó el surgimiento y la adopción masiva de los sistemas gestores de bases de datos relacionales, con énfasis en la arquitectura de Oracle Database y su comparación con PostgreSQL, MySQL (InnoDB) y Microsoft SQL Server. A partir de un recorrido histórico que va desde los sistemas de archivos planos y los modelos jerárquico y en red hasta la propuesta relacional de Edgar F. Codd en 1970, se argumenta que la demora en la adopción del modelo relacional respondió menos a limitaciones teóricas que a restricciones de memoria y procesamiento disponibles en su momento.

Con base en una metodología documental y comparativa, el estudio describe la arquitectura interna de Oracle (su organización de memoria, procesos en segundo plano, estructura de archivos físicos y mecanismos para garantizar propiedades ACID) y contrasta sus decisiones de diseño con las de los otros motores analizados. Los resultados muestran que no existe un sistema universalmente superior, sino distintas soluciones a problemas comunes, influenciadas por contextos tecnológicos y filosóficos específicos. Se concluye que comprender la arquitectura interna de un SGBD es fundamental para diseñar sistemas escalables y resilientes, más allá de una perspectiva meramente teórica.

\bigskip
\noindent\textbf{Palabras clave:} bases de datos relacionales, Oracle Database,
arquitectura SGBD, memoria RAM, PostgreSQL, MySQL, SQL Server, Edgar F. Codd.
\clearpage